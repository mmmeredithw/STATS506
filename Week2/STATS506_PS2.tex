% Options for packages loaded elsewhere
\PassOptionsToPackage{unicode}{hyperref}
\PassOptionsToPackage{hyphens}{url}
\PassOptionsToPackage{dvipsnames,svgnames,x11names}{xcolor}
%
\documentclass[
  letterpaper,
  DIV=11,
  numbers=noendperiod]{scrartcl}

\usepackage{amsmath,amssymb}
\usepackage{iftex}
\ifPDFTeX
  \usepackage[T1]{fontenc}
  \usepackage[utf8]{inputenc}
  \usepackage{textcomp} % provide euro and other symbols
\else % if luatex or xetex
  \usepackage{unicode-math}
  \defaultfontfeatures{Scale=MatchLowercase}
  \defaultfontfeatures[\rmfamily]{Ligatures=TeX,Scale=1}
\fi
\usepackage{lmodern}
\ifPDFTeX\else  
    % xetex/luatex font selection
\fi
% Use upquote if available, for straight quotes in verbatim environments
\IfFileExists{upquote.sty}{\usepackage{upquote}}{}
\IfFileExists{microtype.sty}{% use microtype if available
  \usepackage[]{microtype}
  \UseMicrotypeSet[protrusion]{basicmath} % disable protrusion for tt fonts
}{}
\makeatletter
\@ifundefined{KOMAClassName}{% if non-KOMA class
  \IfFileExists{parskip.sty}{%
    \usepackage{parskip}
  }{% else
    \setlength{\parindent}{0pt}
    \setlength{\parskip}{6pt plus 2pt minus 1pt}}
}{% if KOMA class
  \KOMAoptions{parskip=half}}
\makeatother
\usepackage{xcolor}
\setlength{\emergencystretch}{3em} % prevent overfull lines
\setcounter{secnumdepth}{-\maxdimen} % remove section numbering
% Make \paragraph and \subparagraph free-standing
\ifx\paragraph\undefined\else
  \let\oldparagraph\paragraph
  \renewcommand{\paragraph}[1]{\oldparagraph{#1}\mbox{}}
\fi
\ifx\subparagraph\undefined\else
  \let\oldsubparagraph\subparagraph
  \renewcommand{\subparagraph}[1]{\oldsubparagraph{#1}\mbox{}}
\fi

\usepackage{color}
\usepackage{fancyvrb}
\newcommand{\VerbBar}{|}
\newcommand{\VERB}{\Verb[commandchars=\\\{\}]}
\DefineVerbatimEnvironment{Highlighting}{Verbatim}{commandchars=\\\{\}}
% Add ',fontsize=\small' for more characters per line
\usepackage{framed}
\definecolor{shadecolor}{RGB}{241,243,245}
\newenvironment{Shaded}{\begin{snugshade}}{\end{snugshade}}
\newcommand{\AlertTok}[1]{\textcolor[rgb]{0.68,0.00,0.00}{#1}}
\newcommand{\AnnotationTok}[1]{\textcolor[rgb]{0.37,0.37,0.37}{#1}}
\newcommand{\AttributeTok}[1]{\textcolor[rgb]{0.40,0.45,0.13}{#1}}
\newcommand{\BaseNTok}[1]{\textcolor[rgb]{0.68,0.00,0.00}{#1}}
\newcommand{\BuiltInTok}[1]{\textcolor[rgb]{0.00,0.23,0.31}{#1}}
\newcommand{\CharTok}[1]{\textcolor[rgb]{0.13,0.47,0.30}{#1}}
\newcommand{\CommentTok}[1]{\textcolor[rgb]{0.37,0.37,0.37}{#1}}
\newcommand{\CommentVarTok}[1]{\textcolor[rgb]{0.37,0.37,0.37}{\textit{#1}}}
\newcommand{\ConstantTok}[1]{\textcolor[rgb]{0.56,0.35,0.01}{#1}}
\newcommand{\ControlFlowTok}[1]{\textcolor[rgb]{0.00,0.23,0.31}{#1}}
\newcommand{\DataTypeTok}[1]{\textcolor[rgb]{0.68,0.00,0.00}{#1}}
\newcommand{\DecValTok}[1]{\textcolor[rgb]{0.68,0.00,0.00}{#1}}
\newcommand{\DocumentationTok}[1]{\textcolor[rgb]{0.37,0.37,0.37}{\textit{#1}}}
\newcommand{\ErrorTok}[1]{\textcolor[rgb]{0.68,0.00,0.00}{#1}}
\newcommand{\ExtensionTok}[1]{\textcolor[rgb]{0.00,0.23,0.31}{#1}}
\newcommand{\FloatTok}[1]{\textcolor[rgb]{0.68,0.00,0.00}{#1}}
\newcommand{\FunctionTok}[1]{\textcolor[rgb]{0.28,0.35,0.67}{#1}}
\newcommand{\ImportTok}[1]{\textcolor[rgb]{0.00,0.46,0.62}{#1}}
\newcommand{\InformationTok}[1]{\textcolor[rgb]{0.37,0.37,0.37}{#1}}
\newcommand{\KeywordTok}[1]{\textcolor[rgb]{0.00,0.23,0.31}{#1}}
\newcommand{\NormalTok}[1]{\textcolor[rgb]{0.00,0.23,0.31}{#1}}
\newcommand{\OperatorTok}[1]{\textcolor[rgb]{0.37,0.37,0.37}{#1}}
\newcommand{\OtherTok}[1]{\textcolor[rgb]{0.00,0.23,0.31}{#1}}
\newcommand{\PreprocessorTok}[1]{\textcolor[rgb]{0.68,0.00,0.00}{#1}}
\newcommand{\RegionMarkerTok}[1]{\textcolor[rgb]{0.00,0.23,0.31}{#1}}
\newcommand{\SpecialCharTok}[1]{\textcolor[rgb]{0.37,0.37,0.37}{#1}}
\newcommand{\SpecialStringTok}[1]{\textcolor[rgb]{0.13,0.47,0.30}{#1}}
\newcommand{\StringTok}[1]{\textcolor[rgb]{0.13,0.47,0.30}{#1}}
\newcommand{\VariableTok}[1]{\textcolor[rgb]{0.07,0.07,0.07}{#1}}
\newcommand{\VerbatimStringTok}[1]{\textcolor[rgb]{0.13,0.47,0.30}{#1}}
\newcommand{\WarningTok}[1]{\textcolor[rgb]{0.37,0.37,0.37}{\textit{#1}}}

\providecommand{\tightlist}{%
  \setlength{\itemsep}{0pt}\setlength{\parskip}{0pt}}\usepackage{longtable,booktabs,array}
\usepackage{calc} % for calculating minipage widths
% Correct order of tables after \paragraph or \subparagraph
\usepackage{etoolbox}
\makeatletter
\patchcmd\longtable{\par}{\if@noskipsec\mbox{}\fi\par}{}{}
\makeatother
% Allow footnotes in longtable head/foot
\IfFileExists{footnotehyper.sty}{\usepackage{footnotehyper}}{\usepackage{footnote}}
\makesavenoteenv{longtable}
\usepackage{graphicx}
\makeatletter
\def\maxwidth{\ifdim\Gin@nat@width>\linewidth\linewidth\else\Gin@nat@width\fi}
\def\maxheight{\ifdim\Gin@nat@height>\textheight\textheight\else\Gin@nat@height\fi}
\makeatother
% Scale images if necessary, so that they will not overflow the page
% margins by default, and it is still possible to overwrite the defaults
% using explicit options in \includegraphics[width, height, ...]{}
\setkeys{Gin}{width=\maxwidth,height=\maxheight,keepaspectratio}
% Set default figure placement to htbp
\makeatletter
\def\fps@figure{htbp}
\makeatother

\KOMAoption{captions}{tableheading}
\makeatletter
\makeatother
\makeatletter
\makeatother
\makeatletter
\@ifpackageloaded{caption}{}{\usepackage{caption}}
\AtBeginDocument{%
\ifdefined\contentsname
  \renewcommand*\contentsname{Table of contents}
\else
  \newcommand\contentsname{Table of contents}
\fi
\ifdefined\listfigurename
  \renewcommand*\listfigurename{List of Figures}
\else
  \newcommand\listfigurename{List of Figures}
\fi
\ifdefined\listtablename
  \renewcommand*\listtablename{List of Tables}
\else
  \newcommand\listtablename{List of Tables}
\fi
\ifdefined\figurename
  \renewcommand*\figurename{Figure}
\else
  \newcommand\figurename{Figure}
\fi
\ifdefined\tablename
  \renewcommand*\tablename{Table}
\else
  \newcommand\tablename{Table}
\fi
}
\@ifpackageloaded{float}{}{\usepackage{float}}
\floatstyle{ruled}
\@ifundefined{c@chapter}{\newfloat{codelisting}{h}{lop}}{\newfloat{codelisting}{h}{lop}[chapter]}
\floatname{codelisting}{Listing}
\newcommand*\listoflistings{\listof{codelisting}{List of Listings}}
\makeatother
\makeatletter
\@ifpackageloaded{caption}{}{\usepackage{caption}}
\@ifpackageloaded{subcaption}{}{\usepackage{subcaption}}
\makeatother
\makeatletter
\@ifpackageloaded{tcolorbox}{}{\usepackage[skins,breakable]{tcolorbox}}
\makeatother
\makeatletter
\@ifundefined{shadecolor}{\definecolor{shadecolor}{rgb}{.97, .97, .97}}
\makeatother
\makeatletter
\makeatother
\makeatletter
\makeatother
\ifLuaTeX
  \usepackage{selnolig}  % disable illegal ligatures
\fi
\IfFileExists{bookmark.sty}{\usepackage{bookmark}}{\usepackage{hyperref}}
\IfFileExists{xurl.sty}{\usepackage{xurl}}{} % add URL line breaks if available
\urlstyle{same} % disable monospaced font for URLs
\hypersetup{
  pdftitle={STATS506\_PS2},
  colorlinks=true,
  linkcolor={blue},
  filecolor={Maroon},
  citecolor={Blue},
  urlcolor={Blue},
  pdfcreator={LaTeX via pandoc}}

\title{STATS506\_PS2}
\author{}
\date{}

\begin{document}
\maketitle
\ifdefined\Shaded\renewenvironment{Shaded}{\begin{tcolorbox}[enhanced, interior hidden, frame hidden, borderline west={3pt}{0pt}{shadecolor}, breakable, sharp corners, boxrule=0pt]}{\end{tcolorbox}}\fi

\begin{Shaded}
\begin{Highlighting}[]
\FunctionTok{library}\NormalTok{(microbenchmark)}
\FunctionTok{library}\NormalTok{(dplyr)}
\end{Highlighting}
\end{Shaded}

\begin{verbatim}

Attaching package: 'dplyr'
\end{verbatim}

\begin{verbatim}
The following objects are masked from 'package:stats':

    filter, lag
\end{verbatim}

\begin{verbatim}
The following objects are masked from 'package:base':

    intersect, setdiff, setequal, union
\end{verbatim}

\begin{Shaded}
\begin{Highlighting}[]
\FunctionTok{library}\NormalTok{(ggplot2)}
\end{Highlighting}
\end{Shaded}

Question1

(a)

\begin{Shaded}
\begin{Highlighting}[]
\NormalTok{play\_dice1 }\OtherTok{\textless{}{-}} \ControlFlowTok{function}\NormalTok{(n) \{}
\NormalTok{  total }\OtherTok{\textless{}{-}} \DecValTok{0}
  \ControlFlowTok{for}\NormalTok{ (i }\ControlFlowTok{in} \DecValTok{1}\SpecialCharTok{:}\NormalTok{n) \{}
\NormalTok{    roll }\OtherTok{\textless{}{-}} \FunctionTok{sample}\NormalTok{(}\DecValTok{1}\SpecialCharTok{:}\DecValTok{6}\NormalTok{, }\DecValTok{1}\NormalTok{)}
    \ControlFlowTok{if}\NormalTok{ (roll }\SpecialCharTok{\%\%} \DecValTok{2} \SpecialCharTok{==} \DecValTok{0}\NormalTok{) total }\OtherTok{\textless{}{-}}\NormalTok{ total }\SpecialCharTok{+}\NormalTok{ roll }\SpecialCharTok{{-}}\DecValTok{2}
    \ControlFlowTok{else}\NormalTok{ total }\OtherTok{\textless{}{-}}\NormalTok{ total }\SpecialCharTok{{-}} \DecValTok{2}
\NormalTok{  \}}
  \FunctionTok{return}\NormalTok{(total)}
\NormalTok{\}}
\end{Highlighting}
\end{Shaded}

(b)

\begin{Shaded}
\begin{Highlighting}[]
\NormalTok{play\_dice2 }\OtherTok{\textless{}{-}} \ControlFlowTok{function}\NormalTok{(n)\{}
\NormalTok{  roll }\OtherTok{\textless{}{-}} \FunctionTok{sample}\NormalTok{(}\DecValTok{1}\SpecialCharTok{:}\DecValTok{6}\NormalTok{, n, }\AttributeTok{replace =} \ConstantTok{TRUE}\NormalTok{)}
\NormalTok{  win }\OtherTok{\textless{}{-}} \FunctionTok{sum}\NormalTok{(roll[roll }\SpecialCharTok{\%\%} \DecValTok{2} \SpecialCharTok{==} \DecValTok{0}\NormalTok{])}\SpecialCharTok{+}\FunctionTok{length}\NormalTok{(roll[roll }\SpecialCharTok{\%\%} \DecValTok{2} \SpecialCharTok{==} \DecValTok{0}\NormalTok{]) }\SpecialCharTok{*}\NormalTok{ (}\SpecialCharTok{{-}}\DecValTok{2}\NormalTok{)}
\NormalTok{  lose }\OtherTok{\textless{}{-}} \FunctionTok{length}\NormalTok{(roll[roll }\SpecialCharTok{\%\%} \DecValTok{2} \SpecialCharTok{!=} \DecValTok{0}\NormalTok{]) }\SpecialCharTok{*}\NormalTok{ (}\SpecialCharTok{{-}}\DecValTok{2}\NormalTok{)}
  \FunctionTok{return}\NormalTok{(win }\SpecialCharTok{+}\NormalTok{ lose)}
\NormalTok{\}}
\end{Highlighting}
\end{Shaded}

(c)

\begin{Shaded}
\begin{Highlighting}[]
\NormalTok{play\_dice3 }\OtherTok{\textless{}{-}} \ControlFlowTok{function}\NormalTok{(n)\{}
\NormalTok{  roll }\OtherTok{\textless{}{-}} \FunctionTok{sample}\NormalTok{(}\DecValTok{1}\SpecialCharTok{:}\DecValTok{6}\NormalTok{, n, }\AttributeTok{replace =} \ConstantTok{TRUE}\NormalTok{)}
\NormalTok{  roll\_to\_factor }\OtherTok{\textless{}{-}} \FunctionTok{factor}\NormalTok{(roll, }\AttributeTok{levels =} \DecValTok{1}\SpecialCharTok{:}\DecValTok{6}\NormalTok{)}
\NormalTok{  roll\_table }\OtherTok{\textless{}{-}} \FunctionTok{table}\NormalTok{(roll\_to\_factor)}
\NormalTok{  win }\OtherTok{\textless{}{-}} \FunctionTok{sum}\NormalTok{((}\DecValTok{1}\SpecialCharTok{:}\DecValTok{6}\NormalTok{)[(}\DecValTok{1}\SpecialCharTok{:}\DecValTok{6}\NormalTok{) }\SpecialCharTok{\%\%} \DecValTok{2} \SpecialCharTok{==} \DecValTok{0}\NormalTok{]}\SpecialCharTok{*}\NormalTok{roll\_table[(}\DecValTok{1}\SpecialCharTok{:}\DecValTok{6}\NormalTok{) }\SpecialCharTok{\%\%} \DecValTok{2} \SpecialCharTok{==} \DecValTok{0}\NormalTok{])}\SpecialCharTok{+}\FunctionTok{sum}\NormalTok{(roll\_table[(}\DecValTok{1}\SpecialCharTok{:}\DecValTok{6}\NormalTok{) }\SpecialCharTok{\%\%} \DecValTok{2} \SpecialCharTok{==} \DecValTok{0}\NormalTok{])}\SpecialCharTok{*}\NormalTok{(}\SpecialCharTok{{-}}\DecValTok{2}\NormalTok{)}
\NormalTok{  lose }\OtherTok{\textless{}{-}} \FunctionTok{sum}\NormalTok{(roll\_table[(}\DecValTok{1}\SpecialCharTok{:}\DecValTok{6}\NormalTok{) }\SpecialCharTok{\%\%} \DecValTok{2} \SpecialCharTok{!=} \DecValTok{0}\NormalTok{])}\SpecialCharTok{*}\NormalTok{(}\SpecialCharTok{{-}}\DecValTok{2}\NormalTok{)}
  \FunctionTok{return}\NormalTok{(win}\SpecialCharTok{+}\NormalTok{lose)            }
\NormalTok{\}}
\end{Highlighting}
\end{Shaded}

(d)

\begin{Shaded}
\begin{Highlighting}[]
\NormalTok{play\_dice4 }\OtherTok{\textless{}{-}} \ControlFlowTok{function}\NormalTok{(n)\{}
\NormalTok{  one\_roll }\OtherTok{\textless{}{-}} \ControlFlowTok{function}\NormalTok{(x)\{}
\NormalTok{    roll}\OtherTok{\textless{}{-}}\FunctionTok{sample}\NormalTok{(}\DecValTok{1}\SpecialCharTok{:}\DecValTok{6}\NormalTok{,}\DecValTok{1}\NormalTok{)}
    \ControlFlowTok{if}\NormalTok{ (roll}\SpecialCharTok{\%\%}\DecValTok{2}\SpecialCharTok{==}\DecValTok{0}\NormalTok{)\{}
      \FunctionTok{return}\NormalTok{(roll}\DecValTok{{-}2}\NormalTok{)}
\NormalTok{    \} }\ControlFlowTok{else}\NormalTok{(}
      \FunctionTok{return}\NormalTok{(}\SpecialCharTok{{-}}\DecValTok{2}\NormalTok{)}
\NormalTok{    )}
\NormalTok{  \}}
\NormalTok{  total }\OtherTok{\textless{}{-}} \FunctionTok{sum}\NormalTok{(}\FunctionTok{sapply}\NormalTok{(}\DecValTok{1}\SpecialCharTok{:}\NormalTok{n,one\_roll))}
  \FunctionTok{return}\NormalTok{(total)}
\NormalTok{\}}
\end{Highlighting}
\end{Shaded}

b.

\begin{Shaded}
\begin{Highlighting}[]
\FunctionTok{cat}\NormalTok{(}\StringTok{"play\_dice1 with input 3 "}\NormalTok{, }\FunctionTok{play\_dice1}\NormalTok{(}\DecValTok{3}\NormalTok{), }\StringTok{"}\SpecialCharTok{\textbackslash{}n}\StringTok{"}\NormalTok{)}
\end{Highlighting}
\end{Shaded}

\begin{verbatim}
play_dice1 with input 3  2 
\end{verbatim}

\begin{Shaded}
\begin{Highlighting}[]
\FunctionTok{cat}\NormalTok{(}\StringTok{"play\_dice1 with input 3000 "}\NormalTok{, }\FunctionTok{play\_dice1}\NormalTok{(}\DecValTok{3000}\NormalTok{), }\StringTok{"}\SpecialCharTok{\textbackslash{}n}\StringTok{"}\NormalTok{)}
\end{Highlighting}
\end{Shaded}

\begin{verbatim}
play_dice1 with input 3000  -104 
\end{verbatim}

\begin{Shaded}
\begin{Highlighting}[]
\FunctionTok{cat}\NormalTok{(}\StringTok{"play\_dice2 with input 3 "}\NormalTok{, }\FunctionTok{play\_dice2}\NormalTok{(}\DecValTok{3}\NormalTok{), }\StringTok{"}\SpecialCharTok{\textbackslash{}n}\StringTok{"}\NormalTok{)}
\end{Highlighting}
\end{Shaded}

\begin{verbatim}
play_dice2 with input 3  4 
\end{verbatim}

\begin{Shaded}
\begin{Highlighting}[]
\FunctionTok{cat}\NormalTok{(}\StringTok{"play\_dice2 with input 3000 "}\NormalTok{, }\FunctionTok{play\_dice2}\NormalTok{(}\DecValTok{3000}\NormalTok{), }\StringTok{"}\SpecialCharTok{\textbackslash{}n}\StringTok{"}\NormalTok{)}
\end{Highlighting}
\end{Shaded}

\begin{verbatim}
play_dice2 with input 3000  -20 
\end{verbatim}

\begin{Shaded}
\begin{Highlighting}[]
\FunctionTok{cat}\NormalTok{(}\StringTok{"play\_dice3 with input 3 "}\NormalTok{, }\FunctionTok{play\_dice3}\NormalTok{(}\DecValTok{3}\NormalTok{), }\StringTok{"}\SpecialCharTok{\textbackslash{}n}\StringTok{"}\NormalTok{)}
\end{Highlighting}
\end{Shaded}

\begin{verbatim}
play_dice3 with input 3  -2 
\end{verbatim}

\begin{Shaded}
\begin{Highlighting}[]
\FunctionTok{cat}\NormalTok{(}\StringTok{"play\_dice3 with input 3000 "}\NormalTok{, }\FunctionTok{play\_dice3}\NormalTok{(}\DecValTok{3000}\NormalTok{), }\StringTok{"}\SpecialCharTok{\textbackslash{}n}\StringTok{"}\NormalTok{)}
\end{Highlighting}
\end{Shaded}

\begin{verbatim}
play_dice3 with input 3000  4 
\end{verbatim}

\begin{Shaded}
\begin{Highlighting}[]
\FunctionTok{cat}\NormalTok{(}\StringTok{"play\_dice4 with input 3 "}\NormalTok{, }\FunctionTok{play\_dice4}\NormalTok{(}\DecValTok{3}\NormalTok{), }\StringTok{"}\SpecialCharTok{\textbackslash{}n}\StringTok{"}\NormalTok{)}
\end{Highlighting}
\end{Shaded}

\begin{verbatim}
play_dice4 with input 3  -2 
\end{verbatim}

\begin{Shaded}
\begin{Highlighting}[]
\FunctionTok{cat}\NormalTok{(}\StringTok{"play\_dice4 with input 3000 "}\NormalTok{, }\FunctionTok{play\_dice4}\NormalTok{(}\DecValTok{3000}\NormalTok{), }\StringTok{"}\SpecialCharTok{\textbackslash{}n}\StringTok{"}\NormalTok{)}
\end{Highlighting}
\end{Shaded}

\begin{verbatim}
play_dice4 with input 3000  70 
\end{verbatim}

As shown above, all the methods can run with the input 3 and 3000. Every
time we run the chunk, the result will be in the range.

c.

\begin{Shaded}
\begin{Highlighting}[]
\FunctionTok{set.seed}\NormalTok{(}\DecValTok{123}\NormalTok{)}
\NormalTok{result1 }\OtherTok{\textless{}{-}} \FunctionTok{play\_dice1}\NormalTok{(}\DecValTok{3}\NormalTok{)}

\FunctionTok{set.seed}\NormalTok{(}\DecValTok{123}\NormalTok{)}
\NormalTok{result2 }\OtherTok{\textless{}{-}} \FunctionTok{play\_dice2}\NormalTok{(}\DecValTok{3}\NormalTok{)}

\FunctionTok{set.seed}\NormalTok{(}\DecValTok{123}\NormalTok{)}
\NormalTok{result3 }\OtherTok{\textless{}{-}} \FunctionTok{play\_dice3}\NormalTok{(}\DecValTok{3}\NormalTok{)}

\FunctionTok{set.seed}\NormalTok{(}\DecValTok{123}\NormalTok{)}
\NormalTok{result4 }\OtherTok{\textless{}{-}} \FunctionTok{play\_dice4}\NormalTok{(}\DecValTok{3}\NormalTok{)}

\FunctionTok{cat}\NormalTok{(}\StringTok{"Results with input 3:}\SpecialCharTok{\textbackslash{}n}\StringTok{"}\NormalTok{)}
\end{Highlighting}
\end{Shaded}

\begin{verbatim}
Results with input 3:
\end{verbatim}

\begin{Shaded}
\begin{Highlighting}[]
\FunctionTok{cat}\NormalTok{(}\StringTok{"play\_dice1: "}\NormalTok{, result1, }\StringTok{"}\SpecialCharTok{\textbackslash{}n}\StringTok{"}\NormalTok{)}
\end{Highlighting}
\end{Shaded}

\begin{verbatim}
play_dice1:  0 
\end{verbatim}

\begin{Shaded}
\begin{Highlighting}[]
\FunctionTok{cat}\NormalTok{(}\StringTok{"play\_dice2: "}\NormalTok{, result2, }\StringTok{"}\SpecialCharTok{\textbackslash{}n}\StringTok{"}\NormalTok{)}
\end{Highlighting}
\end{Shaded}

\begin{verbatim}
play_dice2:  0 
\end{verbatim}

\begin{Shaded}
\begin{Highlighting}[]
\FunctionTok{cat}\NormalTok{(}\StringTok{"play\_dice3: "}\NormalTok{, result3, }\StringTok{"}\SpecialCharTok{\textbackslash{}n}\StringTok{"}\NormalTok{)}
\end{Highlighting}
\end{Shaded}

\begin{verbatim}
play_dice3:  0 
\end{verbatim}

\begin{Shaded}
\begin{Highlighting}[]
\FunctionTok{cat}\NormalTok{(}\StringTok{"play\_dice4: "}\NormalTok{, result4, }\StringTok{"}\SpecialCharTok{\textbackslash{}n\textbackslash{}n}\StringTok{"}\NormalTok{)}
\end{Highlighting}
\end{Shaded}

\begin{verbatim}
play_dice4:  0 
\end{verbatim}

\begin{Shaded}
\begin{Highlighting}[]
\CommentTok{\# Repeat the same process with input 3000}
\FunctionTok{set.seed}\NormalTok{(}\DecValTok{123}\NormalTok{)}
\NormalTok{result1 }\OtherTok{\textless{}{-}} \FunctionTok{play\_dice1}\NormalTok{(}\DecValTok{3000}\NormalTok{)}

\FunctionTok{set.seed}\NormalTok{(}\DecValTok{123}\NormalTok{)}
\NormalTok{result2 }\OtherTok{\textless{}{-}} \FunctionTok{play\_dice2}\NormalTok{(}\DecValTok{3000}\NormalTok{)}

\FunctionTok{set.seed}\NormalTok{(}\DecValTok{123}\NormalTok{)}
\NormalTok{result3 }\OtherTok{\textless{}{-}} \FunctionTok{play\_dice3}\NormalTok{(}\DecValTok{3000}\NormalTok{)}

\FunctionTok{set.seed}\NormalTok{(}\DecValTok{123}\NormalTok{)}
\NormalTok{result4 }\OtherTok{\textless{}{-}} \FunctionTok{play\_dice4}\NormalTok{(}\DecValTok{3000}\NormalTok{)}

\FunctionTok{cat}\NormalTok{(}\StringTok{"Results with input 3000:}\SpecialCharTok{\textbackslash{}n}\StringTok{"}\NormalTok{)}
\end{Highlighting}
\end{Shaded}

\begin{verbatim}
Results with input 3000:
\end{verbatim}

\begin{Shaded}
\begin{Highlighting}[]
\FunctionTok{cat}\NormalTok{(}\StringTok{"play\_dice1: "}\NormalTok{, result1, }\StringTok{"}\SpecialCharTok{\textbackslash{}n}\StringTok{"}\NormalTok{)}
\end{Highlighting}
\end{Shaded}

\begin{verbatim}
play_dice1:  -102 
\end{verbatim}

\begin{Shaded}
\begin{Highlighting}[]
\FunctionTok{cat}\NormalTok{(}\StringTok{"play\_dice2: "}\NormalTok{, result2, }\StringTok{"}\SpecialCharTok{\textbackslash{}n}\StringTok{"}\NormalTok{)}
\end{Highlighting}
\end{Shaded}

\begin{verbatim}
play_dice2:  -102 
\end{verbatim}

\begin{Shaded}
\begin{Highlighting}[]
\FunctionTok{cat}\NormalTok{(}\StringTok{"play\_dice3: "}\NormalTok{, result3, }\StringTok{"}\SpecialCharTok{\textbackslash{}n}\StringTok{"}\NormalTok{)}
\end{Highlighting}
\end{Shaded}

\begin{verbatim}
play_dice3:  -102 
\end{verbatim}

\begin{Shaded}
\begin{Highlighting}[]
\FunctionTok{cat}\NormalTok{(}\StringTok{"play\_dice4: "}\NormalTok{, result4, }\StringTok{"}\SpecialCharTok{\textbackslash{}n}\StringTok{"}\NormalTok{)}
\end{Highlighting}
\end{Shaded}

\begin{verbatim}
play_dice4:  -102 
\end{verbatim}

In this question, `set.seed(123)' is used to control the random dice
rolls simulated by the sample function. Since all functions above use
the sample function and we use the same set.seed(), all the results are
the same.

d.

\begin{Shaded}
\begin{Highlighting}[]
\NormalTok{micro\_low }\OtherTok{\textless{}{-}} \FunctionTok{microbenchmark}\NormalTok{(}
  \FunctionTok{play\_dice1}\NormalTok{(}\DecValTok{100}\NormalTok{),}
  \FunctionTok{play\_dice2}\NormalTok{(}\DecValTok{100}\NormalTok{),}
  \FunctionTok{play\_dice3}\NormalTok{(}\DecValTok{100}\NormalTok{),}
  \FunctionTok{play\_dice4}\NormalTok{(}\DecValTok{100}\NormalTok{),}
  \AttributeTok{times =}\NormalTok{ 100L}
\NormalTok{)}
\end{Highlighting}
\end{Shaded}

\begin{verbatim}
Warning in microbenchmark(play_dice1(100), play_dice2(100), play_dice3(100), :
less accurate nanosecond times to avoid potential integer overflows
\end{verbatim}

\begin{Shaded}
\begin{Highlighting}[]
\NormalTok{micro\_large }\OtherTok{\textless{}{-}} \FunctionTok{microbenchmark}\NormalTok{(}
  \FunctionTok{play\_dice1}\NormalTok{(}\DecValTok{10000}\NormalTok{),}
  \FunctionTok{play\_dice2}\NormalTok{(}\DecValTok{10000}\NormalTok{),}
  \FunctionTok{play\_dice3}\NormalTok{(}\DecValTok{10000}\NormalTok{),}
  \FunctionTok{play\_dice4}\NormalTok{(}\DecValTok{10000}\NormalTok{),}
  \AttributeTok{times =}\NormalTok{ 100L}
\NormalTok{)}
\end{Highlighting}
\end{Shaded}

\begin{Shaded}
\begin{Highlighting}[]
\NormalTok{micro\_low}
\end{Highlighting}
\end{Shaded}

\begin{verbatim}
Unit: microseconds
            expr     min       lq      mean  median       uq      max neval
 play_dice1(100) 229.149 243.8885 318.45356 250.100 259.3865 3010.548   100
 play_dice2(100)   7.667   8.7535   9.65386   9.184   9.7990   22.181   100
 play_dice3(100)  31.283  32.8000  36.06524  33.825  36.9205   81.016   100
 play_dice4(100) 263.753 277.8570 288.68428 283.966 296.0405  369.123   100
\end{verbatim}

\begin{Shaded}
\begin{Highlighting}[]
\NormalTok{micro\_large}
\end{Highlighting}
\end{Shaded}

\begin{verbatim}
Unit: microseconds
              expr       min        lq       mean     median         uq
 play_dice1(10000) 25367.889 26648.524 27860.2384 27688.4890 28727.8800
 play_dice2(10000)   435.543   451.410   466.2713   463.7715   477.0145
 play_dice3(10000)   484.333   520.454   571.5412   534.6195   548.3340
 play_dice4(10000) 27995.087 29841.542 31527.4576 31174.2065 31741.1750
       max neval
 33415.082   100
   527.137   100
  2274.393   100
 58231.480   100
\end{verbatim}

e.

\begin{Shaded}
\begin{Highlighting}[]
\FunctionTok{set.seed}\NormalTok{(}\DecValTok{123}\NormalTok{)}
\NormalTok{n\_simulations }\OtherTok{\textless{}{-}} \DecValTok{10}\SpecialCharTok{\^{}}\DecValTok{6}
\NormalTok{results }\OtherTok{\textless{}{-}} \FunctionTok{play\_dice1}\NormalTok{(n\_simulations)}
\NormalTok{average\_result }\OtherTok{\textless{}{-}}\NormalTok{ results }\SpecialCharTok{/}\NormalTok{ n\_simulations}
\FunctionTok{cat}\NormalTok{(}\StringTok{"Average Result per Game: "}\NormalTok{, average\_result, }\StringTok{"}\SpecialCharTok{\textbackslash{}n}\StringTok{"}\NormalTok{)}
\end{Highlighting}
\end{Shaded}

\begin{verbatim}
Average Result per Game:  -0.001116 
\end{verbatim}

Monte Carlo can be used to simulate a large number of games by observing
the average result. Since the average result using Monte Carlo is close
to the expected value for the game which is 0, we can consider this game
is fair.

Question2

1. import the dataset

\begin{Shaded}
\begin{Highlighting}[]
\NormalTok{data }\OtherTok{\textless{}{-}} \FunctionTok{read.csv}\NormalTok{(}\StringTok{\textquotesingle{}cars.csv\textquotesingle{}}\NormalTok{)}
\FunctionTok{head}\NormalTok{(data)}
\end{Highlighting}
\end{Shaded}

\begin{verbatim}
  Dimensions.Height Dimensions.Length Dimensions.Width
1               140               143              202
2               140               143              202
3               140               143              202
4               140               143              202
5               140               143              202
6                91                17               62
  Engine.Information.Driveline               Engine.Information.Engine.Type
1              All-wheel drive         Audi 3.2L 6 cylinder 250hp 236ft-lbs
2            Front-wheel drive Audi 2.0L 4 cylinder 200 hp 207 ft-lbs Turbo
3            Front-wheel drive Audi 2.0L 4 cylinder 200 hp 207 ft-lbs Turbo
4              All-wheel drive Audi 2.0L 4 cylinder 200 hp 207 ft-lbs Turbo
5              All-wheel drive Audi 2.0L 4 cylinder 200 hp 207 ft-lbs Turbo
6              All-wheel drive        Audi 3.2L 6 cylinder 265hp 243 ft-lbs
  Engine.Information.Hybrid Engine.Information.Number.of.Forward.Gears
1                      True                                          6
2                      True                                          6
3                      True                                          6
4                      True                                          6
5                      True                                          6
6                      True                                          6
  Engine.Information.Transmission Fuel.Information.City.mpg
1  6 Speed Automatic Select Shift                        18
2  6 Speed Automatic Select Shift                        22
3                  6 Speed Manual                        21
4  6 Speed Automatic Select Shift                        21
5  6 Speed Automatic Select Shift                        21
6                  6 Speed Manual                        16
  Fuel.Information.Fuel.Type Fuel.Information.Highway.mpg
1                   Gasoline                           25
2                   Gasoline                           28
3                   Gasoline                           30
4                   Gasoline                           28
5                   Gasoline                           28
6                   Gasoline                           27
  Identification.Classification          Identification.ID Identification.Make
1        Automatic transmission           2009 Audi A3 3.2                Audi
2        Automatic transmission      2009 Audi A3 2.0 T AT                Audi
3           Manual transmission         2009 Audi A3 2.0 T                Audi
4        Automatic transmission 2009 Audi A3 2.0 T Quattro                Audi
5        Automatic transmission 2009 Audi A3 2.0 T Quattro                Audi
6           Manual transmission           2009 Audi A5 3.2                Audi
  Identification.Model.Year Identification.Year
1              2009 Audi A3                2009
2              2009 Audi A3                2009
3              2009 Audi A3                2009
4              2009 Audi A3                2009
5              2009 Audi A3                2009
6              2009 Audi A5                2009
  Engine.Information.Engine.Statistics.Horsepower
1                                             250
2                                             200
3                                             200
4                                             200
5                                             200
6                                             265
  Engine.Information.Engine.Statistics.Torque
1                                         236
2                                         207
3                                         207
4                                         207
5                                         207
6                                         243
\end{verbatim}

\begin{Shaded}
\begin{Highlighting}[]
\FunctionTok{colnames}\NormalTok{(data)}
\end{Highlighting}
\end{Shaded}

\begin{verbatim}
 [1] "Dimensions.Height"                              
 [2] "Dimensions.Length"                              
 [3] "Dimensions.Width"                               
 [4] "Engine.Information.Driveline"                   
 [5] "Engine.Information.Engine.Type"                 
 [6] "Engine.Information.Hybrid"                      
 [7] "Engine.Information.Number.of.Forward.Gears"     
 [8] "Engine.Information.Transmission"                
 [9] "Fuel.Information.City.mpg"                      
[10] "Fuel.Information.Fuel.Type"                     
[11] "Fuel.Information.Highway.mpg"                   
[12] "Identification.Classification"                  
[13] "Identification.ID"                              
[14] "Identification.Make"                            
[15] "Identification.Model.Year"                      
[16] "Identification.Year"                            
[17] "Engine.Information.Engine.Statistics.Horsepower"
[18] "Engine.Information.Engine.Statistics.Torque"    
\end{verbatim}

\begin{Shaded}
\begin{Highlighting}[]
\NormalTok{new\_columnName }\OtherTok{\textless{}{-}} \FunctionTok{c}\NormalTok{(}\StringTok{\textquotesingle{}Height\textquotesingle{}}\NormalTok{, }\StringTok{\textquotesingle{}Length\textquotesingle{}}\NormalTok{, }\StringTok{\textquotesingle{}Width\textquotesingle{}}\NormalTok{, }\StringTok{\textquotesingle{}Driveline\textquotesingle{}}\NormalTok{, }\StringTok{\textquotesingle{}Engine\_type\textquotesingle{}}\NormalTok{,}\StringTok{\textquotesingle{}Engine\_hybrid\textquotesingle{}}\NormalTok{,}\StringTok{\textquotesingle{}Forward\_gears\_num\textquotesingle{}}\NormalTok{, }\StringTok{\textquotesingle{}Transmission\textquotesingle{}}\NormalTok{,}\StringTok{\textquotesingle{}City\_mpg\textquotesingle{}}\NormalTok{,}\StringTok{\textquotesingle{}Fuel\_type\textquotesingle{}}\NormalTok{, }\StringTok{\textquotesingle{}Highway\_mpg\textquotesingle{}}\NormalTok{,}\StringTok{\textquotesingle{}Classification\textquotesingle{}}\NormalTok{,}\StringTok{\textquotesingle{}ID\textquotesingle{}}\NormalTok{,}\StringTok{\textquotesingle{}Make\textquotesingle{}}\NormalTok{,}\StringTok{\textquotesingle{}Model\_year\textquotesingle{}}\NormalTok{,}\StringTok{\textquotesingle{}Year\textquotesingle{}}\NormalTok{,}\StringTok{\textquotesingle{}Horsepower\textquotesingle{}}\NormalTok{,}\StringTok{\textquotesingle{}Torque\textquotesingle{}}\NormalTok{)}
\FunctionTok{colnames}\NormalTok{(data) }\OtherTok{\textless{}{-}}\NormalTok{ new\_columnName}
\FunctionTok{colnames}\NormalTok{(data)}
\end{Highlighting}
\end{Shaded}

\begin{verbatim}
 [1] "Height"            "Length"            "Width"            
 [4] "Driveline"         "Engine_type"       "Engine_hybrid"    
 [7] "Forward_gears_num" "Transmission"      "City_mpg"         
[10] "Fuel_type"         "Highway_mpg"       "Classification"   
[13] "ID"                "Make"              "Model_year"       
[16] "Year"              "Horsepower"        "Torque"           
\end{verbatim}

2.

\begin{Shaded}
\begin{Highlighting}[]
\NormalTok{gasoline\_cars }\OtherTok{\textless{}{-}} \FunctionTok{filter}\NormalTok{(data, Fuel\_type }\SpecialCharTok{==} \StringTok{\textquotesingle{}Gasoline\textquotesingle{}}\NormalTok{)}
\end{Highlighting}
\end{Shaded}

3.

\begin{Shaded}
\begin{Highlighting}[]
\NormalTok{gasoline\_cars}\SpecialCharTok{$}\NormalTok{Year }\OtherTok{\textless{}{-}} \FunctionTok{as.factor}\NormalTok{(gasoline\_cars}\SpecialCharTok{$}\NormalTok{Year)}
\NormalTok{model }\OtherTok{\textless{}{-}} \FunctionTok{lm}\NormalTok{(Highway\_mpg }\SpecialCharTok{\textasciitilde{}}\NormalTok{ Horsepower }\SpecialCharTok{+}\NormalTok{ Torque }\SpecialCharTok{+}\NormalTok{ Length }\SpecialCharTok{+}\NormalTok{ Width }\SpecialCharTok{+}\NormalTok{ Height }\SpecialCharTok{+}\NormalTok{ Year, }\AttributeTok{data =}\NormalTok{ gasoline\_cars)}
\FunctionTok{summary}\NormalTok{(model)}
\end{Highlighting}
\end{Shaded}

\begin{verbatim}

Call:
lm(formula = Highway_mpg ~ Horsepower + Torque + Length + Width + 
    Height + Year, data = gasoline_cars)

Residuals:
    Min      1Q  Median      3Q     Max 
-10.824  -2.550  -0.452   2.372 202.639 

Coefficients:
              Estimate Std. Error t value Pr(>|t|)    
(Intercept) 32.2926630  0.7225982  44.690  < 2e-16 ***
Horsepower   0.0163556  0.0022772   7.182 7.96e-13 ***
Torque      -0.0507425  0.0022030 -23.034  < 2e-16 ***
Length       0.0017290  0.0008836   1.957   0.0504 .  
Width       -0.0003343  0.0009045  -0.370   0.7117    
Height       0.0099079  0.0011267   8.794  < 2e-16 ***
Year2010    -0.4539681  0.6768246  -0.671   0.5024    
Year2011     0.1711016  0.6757043   0.253   0.8001    
Year2012     1.3029279  0.6810076   1.913   0.0558 .  
---
Signif. codes:  0 '***' 0.001 '**' 0.01 '*' 0.05 '.' 0.1 ' ' 1

Residual standard error: 4.602 on 4582 degrees of freedom
Multiple R-squared:  0.4192,    Adjusted R-squared:  0.4182 
F-statistic: 413.3 on 8 and 4582 DF,  p-value: < 2.2e-16
\end{verbatim}

We can see that the p-value for Horsepower is less than 0.05, which
gives us that there's sufficient evidence to suggest that the
relationship between horsepower and MPG is statistically significant.
Also, the coefficient gives us that with all others constant, i unit
increase in Horsepower will lead to 0.0164 increase unit Highway\_mpg.

4.

\begin{Shaded}
\begin{Highlighting}[]
\NormalTok{gasoline\_cars}\SpecialCharTok{$}\NormalTok{Year }\OtherTok{\textless{}{-}} \FunctionTok{as.factor}\NormalTok{(gasoline\_cars}\SpecialCharTok{$}\NormalTok{Year)}
\NormalTok{model2 }\OtherTok{\textless{}{-}} \FunctionTok{lm}\NormalTok{(Highway\_mpg }\SpecialCharTok{\textasciitilde{}}\NormalTok{ Horsepower}\SpecialCharTok{*}\NormalTok{Torque }\SpecialCharTok{+}\NormalTok{ Length }\SpecialCharTok{+}\NormalTok{ Width }\SpecialCharTok{+}\NormalTok{ Height }\SpecialCharTok{+}\NormalTok{ Year, }\AttributeTok{data =}\NormalTok{ gasoline\_cars)}
\end{Highlighting}
\end{Shaded}

\begin{Shaded}
\begin{Highlighting}[]
\FunctionTok{set.seed}\NormalTok{(}\DecValTok{123}\NormalTok{)}
\NormalTok{gasoline\_cars}\SpecialCharTok{$}\NormalTok{Year }\OtherTok{\textless{}{-}} \FunctionTok{as.factor}\NormalTok{(gasoline\_cars}\SpecialCharTok{$}\NormalTok{Year)}
\NormalTok{torque\_values }\OtherTok{\textless{}{-}} \FunctionTok{c}\NormalTok{(}\DecValTok{200}\NormalTok{,}\DecValTok{250}\NormalTok{,}\DecValTok{300}\NormalTok{)}
\NormalTok{new\_data }\OtherTok{\textless{}{-}} \FunctionTok{expand.grid}\NormalTok{(}
  \AttributeTok{Horsepower =} \FunctionTok{seq}\NormalTok{(}\FunctionTok{min}\NormalTok{(gasoline\_cars}\SpecialCharTok{$}\NormalTok{Horsepower), }\FunctionTok{max}\NormalTok{(gasoline\_cars}\SpecialCharTok{$}\NormalTok{Horsepower), }\AttributeTok{length.out =} \DecValTok{100}\NormalTok{), }
  \AttributeTok{Torque =}\NormalTok{ torque\_values, }
  \AttributeTok{Length =} \FunctionTok{mean}\NormalTok{(gasoline\_cars}\SpecialCharTok{$}\NormalTok{Length), }
  \AttributeTok{Width =} \FunctionTok{mean}\NormalTok{(gasoline\_cars}\SpecialCharTok{$}\NormalTok{Width),}
  \AttributeTok{Height =} \FunctionTok{mean}\NormalTok{(gasoline\_cars}\SpecialCharTok{$}\NormalTok{Height),}
  \AttributeTok{Year =} \FunctionTok{levels}\NormalTok{(gasoline\_cars}\SpecialCharTok{$}\NormalTok{Year)[}\DecValTok{1}\NormalTok{] }
\NormalTok{)}
\NormalTok{new\_data}\SpecialCharTok{$}\NormalTok{predicted\_mpg }\OtherTok{\textless{}{-}} \FunctionTok{predict}\NormalTok{(model2, }\AttributeTok{newdata =}\NormalTok{ new\_data)}
\FunctionTok{ggplot}\NormalTok{(new\_data, }\FunctionTok{aes}\NormalTok{(}\AttributeTok{x =}\NormalTok{ Horsepower, }\AttributeTok{y =}\NormalTok{ predicted\_mpg, }\AttributeTok{color =} \FunctionTok{as.factor}\NormalTok{(Torque))) }\SpecialCharTok{+}
   \FunctionTok{geom\_line}\NormalTok{() }\SpecialCharTok{+}
   \FunctionTok{labs}\NormalTok{(}\AttributeTok{color =} \StringTok{"Torque"}\NormalTok{) }\SpecialCharTok{+}
   \FunctionTok{ggtitle}\NormalTok{(}\StringTok{"Interaction Plot: Horsepower and Torque"}\NormalTok{) }\SpecialCharTok{+}
   \FunctionTok{xlab}\NormalTok{(}\StringTok{"Horsepower"}\NormalTok{) }\SpecialCharTok{+}
   \FunctionTok{ylab}\NormalTok{(}\StringTok{"Predicted MPG"}\NormalTok{)}
\end{Highlighting}
\end{Shaded}

\begin{figure}[H]

{\centering \includegraphics{STATS506_PS2_files/figure-pdf/unnamed-chunk-19-1.pdf}

}

\end{figure}

e.

\begin{Shaded}
\begin{Highlighting}[]
\NormalTok{data}\SpecialCharTok{$}\NormalTok{Year }\OtherTok{\textless{}{-}} \FunctionTok{as.factor}\NormalTok{(data}\SpecialCharTok{$}\NormalTok{Year)}
\NormalTok{X }\OtherTok{\textless{}{-}} \FunctionTok{model.matrix}\NormalTok{(Highway\_mpg }\SpecialCharTok{\textasciitilde{}}\NormalTok{ Horsepower }\SpecialCharTok{+}\NormalTok{ Torque }\SpecialCharTok{+}\NormalTok{ Length }\SpecialCharTok{+}\NormalTok{ Width }\SpecialCharTok{+}\NormalTok{ Height }\SpecialCharTok{+}\NormalTok{ Year, }\AttributeTok{data =}\NormalTok{ data)}
\NormalTok{Y }\OtherTok{\textless{}{-}}\NormalTok{ data}\SpecialCharTok{$}\NormalTok{Highway\_mpg}
\NormalTok{beta\_hat }\OtherTok{\textless{}{-}} \FunctionTok{solve}\NormalTok{(}\FunctionTok{t}\NormalTok{(X) }\SpecialCharTok{\%*\%}\NormalTok{ X) }\SpecialCharTok{\%*\%} \FunctionTok{t}\NormalTok{(X) }\SpecialCharTok{\%*\%}\NormalTok{ Y}
\FunctionTok{print}\NormalTok{(beta\_hat)}
\end{Highlighting}
\end{Shaded}

\begin{verbatim}
                    [,1]
(Intercept) 33.356442255
Horsepower   0.012399025
Torque      -0.048995141
Length       0.000373862
Width       -0.004065288
Height       0.012570213
Year2010    -1.106441206
Year2011    -0.657665240
Year2012     0.566303132
\end{verbatim}

\begin{Shaded}
\begin{Highlighting}[]
\CommentTok{\# Fit the model using lm()}
\NormalTok{data}\SpecialCharTok{$}\NormalTok{Year }\OtherTok{\textless{}{-}} \FunctionTok{as.factor}\NormalTok{(data}\SpecialCharTok{$}\NormalTok{Year)}
\NormalTok{model }\OtherTok{\textless{}{-}} \FunctionTok{lm}\NormalTok{(City\_mpg }\SpecialCharTok{\textasciitilde{}}\NormalTok{ Horsepower }\SpecialCharTok{+}\NormalTok{ Torque }\SpecialCharTok{+}\NormalTok{ Length }\SpecialCharTok{+}\NormalTok{ Width }\SpecialCharTok{+}\NormalTok{ Height }\SpecialCharTok{+}\NormalTok{ Year, }\AttributeTok{data =}\NormalTok{ data)}

\CommentTok{\# Extract coefficients from the lm model}
\NormalTok{lm\_beta }\OtherTok{\textless{}{-}} \FunctionTok{coef}\NormalTok{(model)}

\CommentTok{\# Print the coefficients from lm() model}
\FunctionTok{print}\NormalTok{(lm\_beta)}
\end{Highlighting}
\end{Shaded}

\begin{verbatim}
  (Intercept)    Horsepower        Torque        Length         Width 
 2.594636e+01  5.339315e-04 -3.313173e-02 -9.738097e-05 -3.573970e-04 
       Height      Year2010      Year2011      Year2012 
 8.123714e-03 -1.249538e+00 -9.882162e-01 -2.700390e-01 
\end{verbatim}

We can see that the results from both methods are matched.



\end{document}
